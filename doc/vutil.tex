\documentclass{article}

% Input packages & formatting
\input{template/packages}
\input{template/formatting}
\newcommand{\version}{0.14}


% Other macros

\title{\Huge Tcl Variable Utilities\\\small Version \version}
\author{Alex Baker\\\small\url{https://github.com/ambaker1/vutil}}
\date{\small\today}
\begin{document}
\maketitle
\begin{abstract}
\begin{center}
The ``vutil'' package provides utilities such as read-only variables and TclOO garbage collection.

This package is also a \textcolor{blue}{\href{https://github.com/ambaker1/Tin}{Tin}} package, and can be loaded in as shown below:
\end{center}
\begin{example}{Installing and loading ``vutil''}
\begin{lstlisting}
package require tin
tin add -auto vutil https://github.com/ambaker1/vutil install.tcl 3.0-
tin import vutil
\end{lstlisting}
\end{example}
\end{abstract}

\clearpage
\section{Default Variable Values}
The command \cmdlink{default} assigns a default value to a variable if it does not exist.
\begin{syntax}
\command{default} \$varName \$value
\end{syntax}
\begin{args}
\$varName & Name of variable to set \\
\$value & Default value for variable
\end{args}

The example below shows how default values are only applied if the variable does not exist.
\begin{example}{Variable defaults}
\begin{lstlisting}
set a 5
default a 7; # equivalent to "if {![info exists a]} {set a 7}"
puts $a
unset a
default a 7
puts $a
\end{lstlisting}
\tcblower
\begin{lstlisting}
5
7
\end{lstlisting}
\end{example}
\clearpage
\section{Read-Only Variables}
The command \cmdlink{lock} uses Tcl variable traces to make a read-only variable. 
If attempting to modify a locked variable, it will throw a warning, but not an error.

\begin{syntax}
\command{lock} \$varName <\$value>
\end{syntax}
\begin{args}
\$varName & Variable name to lock.  \\
\$value & Value to lock variable at. Default self-locks (uses current value).
\end{args}

The command \cmdlink{unlock} unlocks previously locked variables so that they can be modified again.
\begin{syntax}
\command{unlock} \$name1 \$name2 …
\end{syntax}
\begin{args}
\$name1 \$name2 … & Variables to unlock.
\end{args}

\begin{example}{Variable locks}
\begin{lstlisting}
lock a 5
set a 7; # throws warning to stderr channel
puts $a
unlock a
set a 7
puts $a
\end{lstlisting}
\tcblower
\begin{lstlisting}
failed to modify "a": read-only
5
7
\end{lstlisting}
\end{example}

Note: You can lock array elements, but not an entire array.

\clearpage

\section{Variable-Object Ties}
As of Tcl version 8.6, there is no garbage collection for Tcl objects, they have to be removed manually with the ``destroy'' method. 
The command \cmdlink{tie} is a solution for this problem, using variable traces to destroy the corresponding object when the variable is unset or modified. 
For example, if an object is tied to a local procedure variable, the object will be destroyed when the procedure returns.

\begin{syntax}
\command{tie} \$varName <\$object>
\end{syntax}
\begin{args}
\$varName & Name of variable for garbage collection. \\
\$object & Object to tie variable to. Default self-ties (uses current value).
\end{args}

In similar fashion to \cmdlink{unlock}, tied variables can be untied with the command \cmdlink{untie}.
\begin{syntax}
\command{untie} \$name1 \$name2 …
\end{syntax}
\begin{args}
\$name1 \$name2 … & Variables to untie.
\end{args}

\begin{example}{Variable-object ties}
\begin{lstlisting}
oo::class create foo {
    method sayhello {} {
        puts {hello world}
    }
}
tie a [foo create bar]
set b $a; # object alias
$a sayhello
$b sayhello
unset a; # destroys object
$b sayhello; # throws error
\end{lstlisting}
\tcblower
\begin{lstlisting}
hello world
hello world
invalid command name "::bar"
\end{lstlisting}
\end{example}
Note: You can tie array elements, but not an entire array, and you cannot tie a locked variable.
\clearpage
\section{Garbage Collection Superclass}
The class \cmdlink{::vutil::GC} is a TclOO superclass that includes garbage collection. 
This class is not exported, and not intended for direct use, as it is simply a template for classes with built-in garbage collection, by tying the object to a specified variable using \cmdlink{tie}.
Below is the syntax for the superclass constructor.

\begin{syntax}
\command{::vutil::GC} new \$varName
\end{syntax}
\begin{syntax}
::vutil::GC create \$name \$varName
\end{syntax}
\begin{args}
\$varName & Name of variable for garbage collection. \\
\$name & Name of object (for ``create'' method).
\end{args}
In addition to tying the object to a variable in the constructor, the \cmdlink{::vutil::GC} superclass also provides a public copy method that sets up garbage collection: ``\methodlink[0]{gc}{-{}->}'', which calls the private method \textit{CopyObject}
\begin{syntax}
\method{gc}{-{}->} \$varName
\end{syntax}
\begin{syntax}
my CopyObject \$varName
\end{syntax}
\begin{args}
\$varName & Name of variable for garbage collection.
\end{args}

Below is an example of how this superclass can be used to build garbage collection into a TclOO class. 
\begin{example}{Simple container class}
\begin{lstlisting}
oo::class create value {
    superclass ::vutil::GC
    variable myValue
    constructor {varName {value {}}} {
        set myValue $value
        next $varName
    }
    method set {value} {set myValue $value}
    method value {} {return $myValue}
}
value new x {hello world}; # create new value, tie to x
[$x --> y] set {foo bar}; # copy to y, set y to {foo bar}
puts [$x value]
puts [$y value]
\end{lstlisting}
\tcblower
\begin{lstlisting}
hello world
foo bar
\end{lstlisting}
\end{example}

\section{Container Superclass}
The class \cmdlink{::vutil::Container} is a TclOO superclass, built on-top of the \cmdlink{::vutil::GC} superclass. 
The value is stored in the superclass variable ``self'', and can be accessed with the methods \textit{GetValue} and \textit{SetValue}.
This class is not exported, and not intended for direct use, but rather is a template for container classes like in the example on the previous page. 
Below is the syntax for the superclass constructor.

\begin{syntax}
\command{::vutil::Container} new \$varName <\$value>
\end{syntax}
\begin{syntax}
::vutil::Container create \$name \$varName <\$value>
\end{syntax}
\begin{args}
\$varName & Name of variable for garbage collection. \\
\$value & Value to store in container. Default blank. \\
\$name & Name of object (for ``create'' method).
\end{args}

Calling a container object by itself calls the \textit{GetValue} method, which queries the value in the container.
\begin{syntax}
\$containerObj
\end{syntax}
\begin{syntax}
my GetValue
\end{syntax}

The assignment operator, ``\methodlink[0]{container}{=}'', calls the \textit{SetValue} method, which sets the value in the container.
\begin{syntax}
\method{container}{=} \$value
\end{syntax}
\begin{syntax}
my SetValue \$value
\end{syntax}
\begin{args}
\$value & Value to store in container. \\
\end{args}
The pipe operator, ``\methodlink[0]{container}{|}'', calls the \textit{TempObject} method, which copies the object to a temporary object, evaluates the method, and returns the result, or the temporary object value if the result is the temporary object.
\begin{syntax}
\method{container}{|} \$method \$arg ...
\end{syntax}
\begin{syntax}
my TempObject \$method \$arg ...
\end{syntax}
\begin{args}
\$method & Method to evaluate in temporary object. \\
\$arg ... & Arguments for method.
\end{args}
\clearpage
\begin{example}{Advanced container class}
\begin{lstlisting}
# Create a class for manipulating lists of floating point values
oo::class create vector {
    superclass ::vutil::Container
    variable self; # Access the "self" variable from superclass
    method SetValue {value} {
        # Convert to double
        next [lmap x $value {::tcl::mathfunc::double $x}]
    }
    method print {args} {
        puts {*}$args $self
    }
    method += {value} {
        set self [lmap x $self {expr {$x + $value}}]
        return [self]
    }
    method -= {value} {
        set self [lmap x $self {expr {$x - $value}}]
        return [self]
    }
    method *= {value} {
        set self [lmap x $self {expr {$x * $value}}]
        return [self]
    }
    method /= {value} {
        set self [lmap x $self {expr {$x / $value}}]
        return [self]
    }
    method @ {index args} {
        if {[llength $args] == 0} {
            return [lindex $self $index]
        } elseif {[llength $args] != 2 || [lindex $args 0] ne "="} {
            return -code error "wrong # args: should be\
                    \"[self] @ index ?= value?\""
        }
        lset self $index [::tcl::mathfunc::double [lindex $args 1]]
        return [self]
    }
    export += -= *= /= @
}
vector new x {1 2 3}
puts [$x | += 5]; # perform operation on temp object
[$x += 5] print; # same operation, on main object
puts [$x @ end]; # index into object
\end{lstlisting}
\tcblower
\begin{lstlisting}
6.0 7.0 8.0
6.0 7.0 8.0
8.0
\end{lstlisting}
\end{example}
\end{document}
































